\documentclass[12pt]{article}

\usepackage{sbc-template}

\usepackage{graphicx,url}

\usepackage[brazil]{babel}   
\usepackage[latin1]{inputenc}  

     
\sloppy

\title{Aplica��o do processo �gil Scrum em um ambiente de Pesquisa em um simulador GSM}

\author{Renato Faria Iida\inst{1}, Renato Willi\inst{2} }


\address{Instituto Nokia de Tecnologia\\
  Bras�lia -- DF -- Brasil
\nextinstitute
  Funda��o Universa\\
  Bras�lia -- DF -- Brasil
  \email{renato.iida@indt.org.br, renato.willi@gmail.com}
}

\begin{document} 

\maketitle

\begin{abstract}
Results of the SCRUM methodology to development of new features of a GSM system level simulator for research proposes compared to current methodology used in the research center.
\end{abstract}
     
\begin{resumo} 
\end{resumo}


\section{Introdu��o}
O instituto de pesquisa Indt tem varias linhas de pesquisa e um deles � um simulador din�mico\cite{blueBook} em GSM que foi um dos projeto defindos  para testar o metodologia �gil SCRUM ao ambiente de pesquisa na unidade de Bras�lia para avaliar se isso traria vantagens em rela��o a metodologia corrent. 

\section{Ambiente de desenvolviento}
Essa sess�o define  como era o projeto e como era a metodologia usada para o desenvolvimento desse projeto e quais eram os problemas que eram mais detectados pelos membros participantes do projeto.
\subsection{Metodologia corrente}
A metodologia antiga do desenvolvimento dessa ferramenta era deixar um respons�vel por cada nova funcionalidade. A parte do GSM seria atualizada para a vers�o EGPRS2, originalmente somente um pesquisador iria ficar respons�vel e uma atualiza��o do canal de voz ficaria respons�vel pro apenais um pesquisador. Normalmente a equipe era formada por quatro pesquisadores e cada um ficava com uma funcionalidade.  Isso gerava v�rios problemas. Um era quando uma pessoa da equipe saia do projeto , a funcionalidade atrelada a ele se perdia pois n�o havia um forma de repassar esse conhecimento entre os membros da equipe.
\subsection{Problemas a serem Resolvidos}
O processo anil SCRUM � cada vez mais utilizado para o desenvolvimento de software e com bons resultados. Ele � muito adaptado a troca de escopo de forma natural ao processo. O processo de pesquisa que � o foco do Indt tinha problemas para gerenciar as demandas para o simulador din�mico de rede para atender as necessidades dos clientes.

\section{Processos ageis} 
%\label{sec:firstpage}

Explicacao sobre o processo rapidos
\subsection{SCRUM}
Explicao sobre SCRUM

\section{Implantacao incial do SCRUM no processo de pesquisa}

Implantacao inicial sobre o processo desnvolvidos
\subsection{Vantagens}

\subsection{Desvantagens}

\section{Ciclo de aprimoramento do processo}


\subsection{Vantagens}

\subsection{Desvantagens}


\section{Conclusao}
 



\bibliographystyle{sbc}
\bibliography{bibliografiaTCC}

\end{document}
